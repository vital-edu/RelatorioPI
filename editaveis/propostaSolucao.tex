O produto foi idealizado como sendo um sistema capaz de integrar vários balões à área a ser monitorada,  no estacionamento da faculdade UnB Gama. Neste caso o único fator limitante na distribuição dos balões cativos seria o alcance do sinal de rádio que transmitirá as suas informações ao solo. O sistema de monitoramento (SUM) será disposto em duas partes: balão cativo e estação de solo.

\subsection{Balão cativo}

O balão cativo portará os equipamentos de sensoriamento capazes de promover a vigilância do local. Tal balão  consistirá de uma bexiga, carga útil, cabeamento de sustentação e base de ancoramento. A bexiga é o componente capaz de gerar uma sustentação, força que promove a subida do balão, e que será preenchida com um tipo de gás menos denso que o ar, gerando força para cima.

A carga útil ou \textit{payload}, é a unidade formada por uma estrutura portando todos  os equipamentos eletrônicos que promovem o monitoramento desejado. Já o cabeamento de sustentação consiste em  três cabos para manter o balão ancorado à superfície (se mantém tracionado), e um cabo para  transmissão de energia.

A base de ancoramento consiste de um equipamento eletro-mecânico preso ao terraço dos prédios, possui um rotor tipo carretel acoplado a um motor elétrico, em que o cabo de sustentação fica enrolado. Este equipamento permite a liberação do balão para subir e também o recolhimento deste de volta à superfície.

Toda a transmissão de dados entre o balão e a estação de solo será feita por meio de ondas de rádio em uma faixa de frequências autorizada pelo órgão nacional competente , a ANATEL.

\subsection{Estação de solo}

A estação de solo possui como função a recepção dos vídeos transmitidos pelo balão, de forma que um operador possa efetuar um diagnóstico do que acontece no estacionamento do campus. Tal estação deve ser capaz de estabelecer a recepção de vídeos de todos os balões do sistema SUM operantes na área de análise.

O sistema presente na estação de solo identifica se  está acontecendo alguma ação suspeita. Ao identificar,  aciona os operadores via rádio  para que eles verifiquem o motivo do alerta feito pela estação Figura \ref{img:processo}.


\begin{figure}[H]
	\centering
	\includegraphics[width=0.6\textwidth]{figuras/processo}
	\caption{Funcionamento SUM}
	\label{img:processo}
\end{figure}
