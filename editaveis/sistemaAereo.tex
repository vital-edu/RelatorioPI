
\subsection{O Envelope} % (fold)
\label{sub:o_envelope}

\subsubsection{Gás do Balão}

	Existem duas possibilidades para a escolha do gás do balão cativo, o gás hélio e o gás hidrogênio. A seguir são apresentadas duas tabelas contendo as características físicas dos gases que podem ser escolhidos para o balão. Nas tabelas \ref{tab:caracHelio} e \ref{tab:caracHidro} as características do hélio e do hidrogênio são tiradas da empresa \textit{Gama Gases}.  

	\begin{table}[H]
		\centering
		\begin{tabular}{|c|c|}
			\hline
			\rowcolor[HTML]{FFFFFF} 
			{\color[HTML]{000000} \textbf{Propriedades}}          & {\color[HTML]{000000} \textbf{Valores Numéricos}}                                    \\ \hline
			Densidade absoluta, gás a 101,325kPa a 0 ºC.           & 0,1785 $Kg/m^3$                                                                         \\ \hline
			Densidade crítica                                     & 0,5307 $Kg/m^3$                                                                       \\ \hline
			Densidade relativa, gás a 101,325 kPa a 0 ºC,(ar = 1). & 0,138                                                                                \\ \hline
			Fator crítico de compressibilidade                    & 0,305                                                                                \\ \hline
			Fórmula                                               & 4He                                                                                  \\ \hline
			Massa Molecular                                       & 4,002602                                                                             \\ \hline
			Pressão crítica                                       & \begin{tabular}[c]{@{}c@{}}229 kPa ; 2,29 bar; 33,2 \\ psia;,2,261 atm.\end{tabular} \\ \hline
			Viscosidade, gás a 101,325 kPa a 26,8 ºC.              & 0,02012 mPa x s; 0,02012 cP.                                                         \\ \hline
			Volume específico a 21,1 ºC 101,325 kPa                & 6030,4 dm3/ kg; 96,6 ft3/ Ib                                                         \\ \hline
		\end{tabular}
		\caption{Características do Hélio}
		\label{tab:caracHelio}
	\end{table}


\begin{table}[H]
	\centering
	\begin{tabular}{|c|c|}
		\hline
		\rowcolor[HTML]{FFFFFF} 
		{\color[HTML]{000000} \textbf{Propriedades}}          & {\color[HTML]{000000} \textbf{Valores Numéricos}}                                     \\ \hline
		Densidade absoluta, gás a 101,325kPa a 0ºC.           & 0,08235 $Kg/m^3$                                                                         \\ \hline
		Densidade crítica                                     & 0,0310 $Kg/m^3$                                                                         \\ \hline
		Densidade relativa, gás a 101,325 kPa a 0ºC,(ar = 1). & 0,0695                                                                                \\ \hline
		Fator crítico de compressibilidade                    & 0,305                                                                                 \\ \hline
		Fórmula                                               & H2                                                                                    \\ \hline
		Limites de inflamabilidade no ar.                     & 4,0-75\% (por volume).                                                                \\ \hline
		Massa Molecular                                       & 2,01588                                                                               \\ \hline
		Pressão crítica                                       & \begin{tabular}[c]{@{}c@{}}1297 kPa; 12,97 bar; 188,1 psia;\\ 12,80 atm.\end{tabular} \\ \hline
		Temperatura de auto-ignição.                          & \begin{tabular}[c]{@{}c@{}}844,3 K; 571,2 ºC; \\ 1060 ºF.\end{tabular}      \\ \hline
		Volume específico a 21,1 ºC, 101,325kPa.         & 11967,4dm3/kg; 191,7ft3/lb.                                                           \\ \hline
	\end{tabular}
	\caption{Características do Hidrogênio}
	\label{tab:caracHidro}
\end{table}

	O gás hidrogênio a primeira vista é mais vantajoso pois é mais leve que o hélio, sua densidade relativa ao ar é de 0.0695 enquanto que a do hélio é de 0.138, e apresentam um fator crítico de compressibilidade iguais. Porém o hidrogênio possui a característica de ser inflamável, enquanto que o hélio é conhecido por ser um gás inerte. Tendo em vista a segurança dos usuários do estacionamento e dos funcionários responsáveis pela manutenção do balão, a exposição ao sol e a possíveis, porém improváveis,  descargas elétricas o hélio se mostra a opção mais vantajosa.

	\subsubsection{Material do Envelope}

	Segundo Yajima (2009), a maioria dos balões atmosféricos são feitos de filme de polietileno. A espessura dos envelopes dos balões usados pela NASA variam de 7 a 90 micrometros dependendo da altitude, funcionalidade, tempo de atividade e peso da payload e etc. Desta forma, cabe analisar que tipo de polietileno será utilizado para a confecção do envelope.
	
	Duas opções de polietileno foram analisadas, o Polietileno Linear de Baixa Densidade (PELBD) e o Polietileno de Baixa Densidade (PEBD). De acordo com Coutinho (ANO), a temperatura máxima de atuação do PELBD é cerca de 120 ºC, sua massa específica varia numa faixa de 0.92 a 094 $g/cm^3$ e possui uma resistência à tração de 37 Mpa.

	A outra opção, o PEBD, segundo Coutinho (ANO), trabalha a uma temperatura máxima de 110 ºC, possui uma massa específica de 0.92 $g/cm^3$ e resistência à tração de 24 Mpa.
	
	Analisando as duas opções de polietileno, o PELBD é o material mais adequado ao envelope do balão, pois possui melhor resistência mecânica. O PEBLD é um termoplástico com elevada capacidade de selagem a quente. É utilizado em filmes para uso industrial, fraldas descartáveis e absorventes, lonas em geral, brinquedos, artigos farmacêuticos e hospitalares, revestimento de fios e cabos.

% subsection o_envelope (end)

\subsection{Modelo do Balão} % (fold)
\label{sub:modelo_do_bal_o}

% subsection modelo_do_bal_o (end)