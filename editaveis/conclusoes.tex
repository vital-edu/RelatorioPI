Até o presente momento foram concluídas as atividades propostas para apresentação nos pontos de controle 1 e 2 da disciplina de Projeto Integrador 1. Dentre as principais atividades desenvolvidas, a definição do escopo, metodologia a ser utilizada, estabelecimento de cronograma, utilização de modelos esquemáticos e integração das soluções das grandes áreas de pesquisas, com utilização de cálculos, foram concluídas. Contudo, se necessário, durante o período de pesquisas do Ponto de Controle 2 para o Ponto de Controle 3, mudanças para o melhoramento do projeto poderão ser feitas.

Os modelos esquemáticos no relatório do PC2 foram desenvolvidos para a melhor visualização da solução elaborada pelos grupos de pesquisa. Por exemplo, o processo de visão geral do produto, o funcionamento da eletrônica embarcada no SUM,  operação da estação de solo. A elaboração de modelos feitos no software CATIA também permitem o leitor identificar os componentes de ancoragem, sustentação e bexiga do balão cativo com melhor detalhamento. Foram feitos também os cálculos de empuxo líquido, forças atuantes no balão, volume, consumo energético.

Dessa forma, com todos os materiais definidos, espera-se para o Ponto de Controle 3 que sejam apresentados os cálculos de viabilidade econômica do projeto, e que as pesquisas sigam o cronograma elaborado, disponível no subtópico 1.5.
